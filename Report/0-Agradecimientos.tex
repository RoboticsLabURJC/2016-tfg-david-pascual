%\chapter*{}
%\pagenumbering{Roman} % para comenzar la numeracion de paginas en numeros romanos
%\begin{flushright}
%	\textit{``I'm afraid that the following syllogism may be used by some in the future:\\	Turing believes machines think.\\	Turing lies with men.\\	Therefore machines do not think."\\  —Alan Turing, 1952}
%\end{flushright}

\chapter*{Agradecimientos}
Este trabajo de fin grado supone para mí la culminación de una dura etapa de aprendizaje de la que, sin duda alguna, he salido reforzado. Reforzado no sólo por los conocimientos adquiridos, si no por las herramientas brindadas para adquirir aquellos que están por llegar. Hasta hace poco desconocía la existencia de materias que a día de hoy me apasionan profundamente. Me siento privilegiado por tener la oportunidad de ser espectador de los cambios que estas materias están produciendo en el mundo, y me emociona trabajar para intentar ser partícipe de ellos. Estos descubrimientos son resultado directo del buen hacer de muchos de los docentes que he conocido estos últimos años. En especial, me gustaría dar las gracias a Inmaculada y Jose María. Su guía incondicional en el desarrollo de este trabajo, lo ha convertido en una experiencia fructífera y gratificante. También quiero dar las gracias a Nuria, compañera de fatigas en estos últimos meses. Gracias a vosotros puedo presumir de estar aprendiendo con los mejores.

Esta aventura ha estado llena de retos y obstáculos, y no tengo ningún reparo en reconocer que jamás habría sido capaz de superar muchos de ellos sin ayuda. En este sentido, quiero dar las gracias a todos los compañeros en los que me he apoyado para superarlos. En particular, agradezco enormemente a Marco y Abel haber compartido conmigo su amistad. Larga vida al \emph{metal}. 

Más alla de lo académico, tengo la fortuna de haber disfrutado de la mejor compañía que pueda imaginar. Por su excelente conversación, su ingenio, su paciencia y su sapiencia, me faltan horas en el día para dar las gracias a Andrés y Alejandra. Os quiero.

Los últimos años, han sido una convulsión de viajes, proyectos, satisfacciones y decepciones. Es difícil digerir estos cambios cuando no encuentras la forma de comunicarte y de hacer comprender a los demás aquello que te aflige. Por suerte, entre esa maraña de cambios, me crucé con Carolina. Su bondad y su forma de ver el mundo es justo aquello que me faltaba para comenzar a entender mis miedos y enfrentarme a ellos. Sigo en proceso. Te quiero.

Por último, doy las gracias a mi familia, y especialmente a mis padres, Virginia y David. Todo lo que he aprendido en la vida puedo dividirlo en dos grandes grupos: aquello que me habéis enseñado vosotros y aquello que me habéis enseñado a aprender.
\chapter*{Summary}
\emph{Object recognition} has been a recurring problem in the history of \emph{computer vision}. Thanks to the inclusion of machine learning and, recently, \emph{deep learning} algorithms, issues like traffic sign recognition and video surveillance have been successfully addressed. In particular, \emph{convolutional neural networks} have become the spearhead of this kind of algorithms in the last few years. In many cases, their effectiveness solving issues like the ones mentioned before can't be denied, which has enabled their usage in commercial applications. Nevertheless, there is some concern about the lack of understanding of their learning process, being accused of acting like \emph{black boxes}.

For all of these reasons, this final degree project aims to reach a deep \emph{understanding} of convolutional neural networks and their \emph{implementation} for solving a certain problem. In this case, a \emph{real-time handwritten digits classifier} will be developed. These objectives will be faced employing \emph{Keras}, a neural networks library for Python, which is well-known because of its simplicity and flexibility. The project starts with the analysis of a convolutional neural network example provided by the aforementioned library. After that, the digit classifier \emph{component} will be discussed. In order to optimize the results achieved by this component, the training data will be processed and new convolutional neural networks with different architectures and learning parameters will be tested. Additionally, a \emph{test bench} will be created. It will be formed by the tools developed both for processing the training data and for computing and visualizing the evaluation parameters. Finally, the performance of the neural networks will be discussed and the one that accomplishes better results will be integrated into the digit classifier component in order to achieve an improved robustness.

The results obtained reveal the great potential of the convolutional neural networks and cast some light on how they are able to learn what they learn, opening the door for their usage in the solution of more complex problems.
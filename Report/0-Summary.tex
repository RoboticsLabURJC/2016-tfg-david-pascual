\chapter*{Summary}
\emph{Object recognition} has been a recurring problem in the history of \emph{computer vision}. Thanks to the inclusion of machine learning and, recently, \emph{deep learning} algorithms, issues like traffic sign recognition and video surveillance have been successfully addressed. In particular, \emph{convolutional neural networks} have become the spearhead of this kind of algorithms in the last few years. In many cases, their effectiveness solving issues like the ones mentioned before can't be denied, which has enabled their usage in commercial applications. Nevertheless, convolutional neural networks keep being accused of acting like \emph{black boxes}, because their learning process is usually very opaque and hard to interpret.

For all of these reasons, this final degree project aims to serve as a \emph{detailed study} of convolutional neural networks and their \emph{implementation} for solving a certain problem. In this case, a \emph{real-time handwritten digits classifier} will be developed. These objectives will be faced employing the \emph{Keras} platform. The project starts with the analysis of a convolutional neural network example provided by the aforementioned library. Then, the digit classifier \emph{component} is discussed. This component acquires images from a video source, classifies them thanks to a neural network built with Keras and displays the result in a graphical user interface.  Besides that, a \emph{test bench} has been created. It is formed by datasets that will feed the convolutional neural networks and tools for computing and visualizing parameters that evaluate their performance. Finally, thanks to the tools developed for the test bench, the effects of the learning process in different neural networks will be discussed, and the one that achieves better results will be integrated within the digit classifier component to accomplish a greater robustness.

The results obtained reveal the great potential of the convolutional neural networks and cast some light on how they are able to learn what they learn, opening the door for their usage in the settlement of more complex problems.
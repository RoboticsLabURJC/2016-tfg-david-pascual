\chapter*{Resumen}
El \emph{reconocimiento de objetos} en imágenes ha sido un problema recurrente en la historia de la \emph{visión artificial}. Gracias a la inclusión de algoritmos basados en aprendizaje máquina y, más recientemente, de las técnicas de \emph{aprendizaje profundo}, se han enfrentado con éxito problemas como el reconocimiento de señales de tráfico o la videovigilancia. En concreto, las \emph{redes neuronales convolucionales} se han convertido en la punta de lanza de este tipo de algoritmos en los últimos años. Su eficacia en la resolución de problemas como los anteriormente mencionados resulta indiscutible en muchos casos, lo que poco a poco está favoreciendo su uso en aplicaciones comerciales. A pesar de ello, existe cierta preocupación por la falta de comprensión sobre su proceso de aprendizaje, siendo acusadas de actuar como una caja negra o \emph{black box}.

Por todo ello, este trabajo de fin grado tiene como metas la \emph{comprensión} de las redes neuronales convolucionales y su \emph{implementación} para la resolución de un determinado problema. En este sentido, se desarrollará un \emph{clasificador de dígitos manuscritos en tiempo real}. Para la consecución de estos objetivos se empleará \emph{Keras}, una librería de redes neuronales para Python que destaca por su sencillez y flexibilidad. El proyecto comienza con el análisis de una red convolucional de ejemplo proporcionada por dicha librería. Posteriormente, se procederá al desarrollo del \emph{componente} clasificador de dígitos. Para optimizar los resultados obtenidos con dicho clasificador, los datos de entrenamiento serán procesados y se entrenarán nuevas redes convolucionales con distintas arquitecturas y condiciones de aprendizaje. Además, se conformorá un \emph{banco de pruebas} en el que se incluirán las herramientas desarrolladas tanto para el procesado de las muestras de entrenamiento, como para el cálculo y visualización de los parámetros de evaluación. Por último, el desempeño de las distintas redes será discutido y la red que mejores resultados arroje será integrada en el componente clasificador de dígitos para lograr una mayor robustez.

Los resultados obtenidos ponen de manifiesto el gran potencial de las redes neuronales convoluciones y proyectan algo de luz sobre cómo son capaces de aprender aquello que aprenden, dejando abierta la puerta a su empleo en la resolución de problemas más complejos.